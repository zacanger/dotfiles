\documentclass{latex2man}

\begin{Name}{1}{iocane}{Jesse McClure}{simulate mouse events}{Iocane - The colorless, oderless, tasteless poison that will eliminate your system's rodent infestation}
\Prog{Iocane}
 - The colorless, oderless, tasteless poison that will eliminate your system's rodent infestation
\end{Name}

\section{Synopsis}
\Prog{iocane}
	\oOpt{-h}
	\oOpt{-i}
	\oOpt{-s}
	\oOptArg{-c}{\ command-string}
	\oArg{filename}

\section{Description}
\Prog{iocane} simulates mouse events from the keyboard, or from commands in a script file.

\section{Options}
\begin{description}
\item[\Opt{-h}]
	Print help and exit
\item[\Opt{-i}]
	Run in interactive mode
\item[\Opt{-s}]
	Run in stdin mode (default)
\item[\OptArg{-c}{\ command-string}]
	Execute command-string.  Multiple commands can be placed on the command line with multiple \Opt{-c} flags.  If \Arg{command-string} contains spaces they must be escaped or quoted.  If any \Opt{-c} commands are specified, \Prog{iocane} will not default to stdin mode and will instead exit after all \Arg{command-string}s are executed.  This can be overridden by explicitly specifying interactive or stdin mode on the command line.
\item[\Arg{filename}]
	Specify an optional \Prog{iocane} script file from which to read commands.
\end{description}

\section{Configuration}
In interactive mode, runtime configuration is read from the following files, in order, stopping at the first file found:
\File{\$XDG_CONFIG_HOME/iocane/config},
\File{\$HOME/.iocanerc},
\File{/usr/share/iocane/iocanerc}.

Each non-empty, non-comment line in a configuration file specifies a \Arg{key} and a \Arg{command-string}.  Any X11 keysym can be used for \Arg{key}.

\section{Commands}

The following \Arg{command-string}s can be entered in the configuration file to be bound to a key, specified as an option to a \Opt{-c} flag on the command line, sent to \Prog{iocane's} stdin, or entered one per line in a script with a #!/bin/iocane shebang.  Any command can also be specified by first letter.

\begin{description}
\item[\Opt{powder}]
	Move the cursor off screen
\item[\Opt{move} \Arg{x} \Arg{y}]
	Move the cursor by (\Arg{x},\Arg{y}) pixels
\item[\Arg{x} \Arg{y}]
	Place the cursor at coordinates (\Arg{x},\Arg{y})
\item[\Opt{button} \Arg{n}]
	Simulate a press of mouse button \Arg{n}
\item[\Opt{cursor} \Arg{n}]
	Sets the cursor to the \Arg{n}th item in X Font Cursors
\item[\Opt{sleep} \Arg{s} \Arg{ms}]
	Sleep for \Arg{s} seconds and \Arg{ms} milliseconds
\item[\Opt{quit} | \Opt{exit}]
	Exit from interactive mode or end a script
\end{description}

\section{Author}
Copyright \copyright 2013 Jesse McClure \\
License GPLv3: GNU GPL version 3 \URL{http://gnu.org/licenses/gpl.html} \\
This is free software: you are free to change and redistribute it. \\
There is NO WARRANTY, to the extent permitted by law.

Submit bug reports via github: \\
\URL{http://github/com/TrilbyWhite/Iocance.git}

I would like your feedback.  If you enjoy \Prog{Iocane} see the bottom of the site below for detauls on submitting comments: \\
\URL{http://mccluresk9.com/software.html}

\LatexManEnd
